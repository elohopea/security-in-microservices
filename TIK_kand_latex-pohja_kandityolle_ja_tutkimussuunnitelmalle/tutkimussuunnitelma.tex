\documentclass[12pt,a4paper,finnish,oneside]{article}

% Valitse 'input encoding':
%\usepackage[latin1]{inputenc} % merkistökoodaus, jos ISO-LATIN-1:tä.
\usepackage[utf8]{inputenc}   % merkistökoodaus, jos käytetään UTF8:a
% Valitse 'output/font encoding':
%\usepackage[T1]{fontenc}      % korjaa ääkkösten tavutusta, bittikarttana
\usepackage{ae,aecompl}       % ed. lis. vektorigrafiikkana bittikartan sijasta
% Kieli- ja tavutuspaketit:
\usepackage[finnish]{babel}
% Muita paketteja:
% \usepackage{amsmath}   % matematiikkaa
\usepackage{url}       % \url{...}

% Kappaleiden erottaminen ja sisennys
\parskip 1ex
\parindent 0pt
\evensidemargin 0mm
\oddsidemargin 0mm
\textwidth 159.2mm
\topmargin 0mm
\headheight 0mm
\headsep 0mm
\textheight 246.2mm

\pagestyle{plain}

% ---------------------------------------------------------------------

\begin{document}

% Otsikkotiedot: muokkaa tähän omat tietosi

\title{TIK.kand tutkimussuunnitelma:\\[5mm]
Turvallisuus mikropalveluarkkitehtuurissa}

\author{Tommi Jäske\\
Aalto-yliopisto\\
\url{tommi.jaske@aalto.fi}}

\date{\today}

\maketitle

% ---------------------------------------------------------------------


\vspace{10mm}

% MUOKKAA TÄHÄN. Jos tarvitset tähän viitteitä, käytä
% tässä dokumentissa numeroviitejärjestelmää komennolla \cite{kahva}.
%
% Paljon kandidaatintöitä ohjanneen Vesa Hirvisalon tarjoama 
% sabluuna. Kursivoidut osat \emph{...} ovat kurssin henkilökunnan
% lisäämiä. 

\textbf{Kandidaatintyön nimi:} Turvallisuus mikropalveluarkkitehtuurissa

\textbf{Työn tekijä:} Tommi Jäske

\textbf{Ohjaaja:} Tuomas Aura


\section{Tiivistelmä tutkimuksesta}

Perinteisesti esimerkiksi verkkopalvelut on tuotettu yhdellä ohjelmalla, 
jota ajetaan serverillä ja yksi ohjelma tuottaa koko palvelun.

Mikropalveluarkkitehtuurissa palvelu on jaettu useisiin osapalveluihin ja 
yhdessä mikropavelut tuottavat käyttäjille heidän tarvitsemiaan kokonaispalveluja.

Mikropalvelut välittävät viestejä toisilleen ja käyttäjälle useimmiten verkon yli. 
Tämä verkko voi olla julkinen tai yksityinen.
Usein yksittäinen mikropalvelu omaa oman tietokantansa, 
joka sisältää tiedot joita mikropalvelu tarvitsee toimiakseen.

Kandidaatin työssä on tarkoitus selvittää korkealla tasolla mikropalveluarkkitehtuurin 
turvallisuutta ja verrata arkkitehtuurin turvallisuus näkökohtia monoliittiseen palveluarkkitehtuuriin.

\section{Tavoitteet ja näkökulmat}

Kandidaatintyössä on tarkoitus selvittää keskeiset erot arkkitehtuurien välillä ja muodostaa kokonaiskuva 
siirtymisestä monoliittisesta palveluarkkitehtuurista mikropalveluarkkitehtuuriin.

Kandidaatin Työn tavoitteena on muodostaa kirjallisuuskatsauksella käsitys olennaisista turvallisuusnäkökohdista 
ja mahdollisista ratkaisuista siirryttäessä jo olemassa olevasta monoliittisesta arkkitehtuurista mikropalveluarkkitehtuuriin.

\section{Tutkimusmateriaali}

Mikropalveluarkkitehtuuri ja etenkin siihen siirtyminen monoliittisestä arkkitehtuurista on 
käsitykseni mukaan ollut voimakkaan tutkimuksen ja kokeilun kohteena.

%TODO

Millaisen aineiston varaan perustat tutkimuksesi? Arvioi materiaalin
riittävyyttä asetettuihin tavoitteisiin nähden.

Pitää olla siis hieman kuvaa siitä, minkälaisen materiaalin kanssa
ollaan tekemisissä ja mitä sellaisen käsittelyyn tarvitaan (etenkin
siis tarvittavan ajan puolesta; ts. kuinka monta tuntia/minuuttia per
lähde?).

\section{Tutkimusmenetelmät}

%TODO

Miten sen keräät materiaalisi tai saat sen käsiisi? Kuinka käsittelet
sen? Kuinka siitä tulee raportti?

Tavallaisesti kirjallisuustutkimuksen yhteydessä tämä on:
(a) lähderyhmien valinta,
(b) viitteiden ja lähteiden haku,
(c) lähteiden arviointi,
(d) lähteiden lukeminen,
(e) tiedon organisointi,
(f) raportointi.  % (f) tärkeää ettei jää vain lukemiseksi!

Kirjallisuustutkimuksen yleinen menetelmä pitää sovittaa tähän
nimenomaiseen aiheeseen sekä tekijän lähtökohtiin. Kuinka sinä teet
muistiinpanot (että myös kirjoitat etkä pelkästään lue). Eli tälle
pitää hieman miettiä omakohtaista vaiheistusta. Siis nähdä ihan
oikeasti, kuinka sinä saat tutkielman tehtyä.

Ja... raportointi ei ole kirjoittamista vaan jo kirjoitettujen
muistiinpanojen koostamista yhteneväksi teokseksi.

\section{Haasteet}

Yleensä kaikkiin töihin liittyy kompastuskiviä. Ne on syytä tiedostaa
etukäteen. Yhdessä työssä aihe on suurpiirteinen (työn rajaaminen
vaikeaa), toisessa materiaalia on niukasti saatavissa, kolmannessa
taas materiaalia on hukkumiseen asti.  Eli, nämä pitäisi kyseisen
tutkimuksen osalta kirjata ylös, ja nähdä ne myös mahdollisuuksina
(positiivisina haasteina) ei ainostaan esteinä.

\section{Resurssit}

Kuka tätä työtä tekee, kuka ohjaa, jne. Paljonko on käyttää
aikaa. Tarvitaanko muuta? (Onko työssä joku kokeellinen osuus?)

\section{Aikataulu}

Kandidaatin työ kirjoitetaan keväällä 2020.

Työhön käytetään kaksi työpäivää viikossa. Yhteensä 15 tuntia viikossa.

Kandidaatin työn kirjoitus on nyt alkanut ja päättyy 26.4.


Laadi tutkimustyölle ja raportoinnille realistinen aikataulu.
Huolehdi, että suunnitelmasi vastaa kandidaatin tutkielman sekä
seminaarin aikataulua sekä laajuutta.  \emph{Kurssiesitteessä omalle
  kirjoitusprosessille on arvioitu noin 6 op eli 160 tuntia eli noin 4
  viikkoa työtä.}

\begin{tabular}{|p{20mm}|p{30mm}|p{95mm}|}
\hline
Viikko   & Työmäärä   & Tuotos                \\ \hline
4        & 15         & Tutkimussuunnitelma   \\ \hline
5        & 15         & V1 -versio            \\ \hline
6        & 15         & V2 -versio            \\ \hline
7        & 15         & V2 -versio            \\ \hline
8        & 15         & V2 -versio            \\ \hline
9        & 15         & V2 -versio valmis     \\ \hline
10       & 15         & V3 -versio            \\ \hline
11       & 15         & V3 -versio            \\ \hline
12       & 15         & V3 -versio            \\ \hline
13       & 15         & V3 -versio valmis     \\ \hline
14       & 15         & V4 -versio            \\ \hline
15       & 15         & V4 -versio valmis     \\ \hline
16       & 15         & Viimeistely           \\ \hline
17       & 15         & Työ valmis            \\ \hline
yhteensä & 210 tuntia & Kandidaatin työ 1 kpl \\ \hline
\end{tabular}


\section{Esittäminen}

Kandidaatin työn alustava sisällysluettelo:
(1) Tiivistelmä,
(2) Johdanto,
(3) Määritelmät,
(3.1) Turvallisuus,
(3.2) Mikropalveluarkkitehtuuri,
(4) Arkkitehtuurin vaihtaminen,
(5) Käytännön kokemukset,
(5.1) Onnistumiset,
(5.2) Epäonnistuneet arkkitehtuurin muutokset ,
(6) Yhteenveto

%

\emph{Rakenne tarkentuu työn edetessä. Tutkimussuunnitelmaan ei välttämättä tarvita lähdeluetteloa, mutta halutessasi voit sisällyttää tärkeimmät lähteet.}

% ---------------------------------------------------------------------
%
% ÄLÄ MUUTA MITÄÄN TÄÄLTÄ LOPUSTA

% Tässä on käytetty siis numeroviittausjärjestelmää. 
% Toinen hyvin yleinen malli on nimi-vuosi-viittaus.

% \bibliographystyle{plainnat}
\bibliographystyle{finplain}  % suomi
%\bibliographystyle{plain}    % englanti
% Lisää mm. http://amath.colorado.edu/documentation/LaTeX/reference/faq/bibstyles.pdf

% Muutetaan otsikko "Kirjallisuutta" -> "Lähteet"
\renewcommand{\refname}{Lähteet}  % article-tyyppisen

% Määritä bib-tiedoston nimi tähän (eli lahteet.bib ilman bib)
\bibliography{lahteet}

% ---------------------------------------------------------------------

\end{document}
