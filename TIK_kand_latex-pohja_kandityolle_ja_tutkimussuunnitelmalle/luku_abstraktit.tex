% Avainsanojen lista pitää merkitä main.tex-tiedoston kohtaan \KEYWORDS.

\begin{fiabstract}
    \begin{sloppypar}
        Verkkopalvelut on usein toteutettu käyttäen monoliittista 
        arkkitehtuuria. Tämä arkkitehtuuri ei skaalaudu eikä mahdollista 
        ketterää kehitystä. Mikropalveluarkkitehtuurin käyttö mahdollistaa nämä 
        sekä useita muita etuja.
    \end{sloppypar}
    \begin{sloppypar}    
        % (1) aihe, tavoite ja rajaus 
        Tämän kandidaatintyön aiheena on tietoturva siirryttäessä 
        monoliittisesta arkkitehtuurista mikropalveluarkkitehtuuriin 
        verkkopalveluissa. Tämän kandidaatintyön tarkoituksena oli selvittää 
        keskeisimmät tietoturvakysymykset arkkitehtuurin vaihdoksessa ja 
        esittää löydettyihin ongelmakohtiin ratkaisuja.
    \end{sloppypar}
    \begin{sloppypar}
        % (2) aineisto ja menetelmät (erittäin lyhyesti);
        Aineistona käytettiin artikkeleja sekä alan 
        peruskirjallisuutta. Työ toteutettiin kirjallisuustutkimuksena.
    \end{sloppypar}
    \begin{sloppypar}        
        % (3) tulokset (tälle enemmän painoarvoa); 
        Tietoturvahaasteita ovat palvelun sisäinen viestintä, saavutettavuus ja 
        skaalautuvuus, ajoympäristön tietoturva ja tunnistaminen ja 
        valtuuttaminen.
    \end{sloppypar}
    \begin{sloppypar}
        % (4) johtopäätökset (tälle enemmän painoarvoa).
        Työssä havaittiin, että mikropalveluarkkitehtuuriinsiirtymisessä on 
        merkittäviä tietoturvariskejä, jotka tulee huolellisesti eritellä ja 
        ratkaista jokaisessa arkkitehtuurin vaihdoksessa tapauskohtaisesti. 
        Tietoturvan suunnittelu ja toteutus tulee suorittaa suurta 
        huolellisuutta noudattaen ja mahdollisuuksien mukaan valmiita 
        toteutuksia käyttäen.
    \end{sloppypar}
\end{fiabstract}







\begin{enabstract}
    \begin{sloppypar}
        There are many web services in use which were designed and implemented 
        before the onslaught of microservices. These services might benefit 
        from architectural change to microservices. Since the architectures
        are very different the security aspects need to be considered.
    \end{sloppypar}
    \begin{sloppypar}    
        % (1) aihe, tavoite ja rajaus 
        The topic of this bachelor's thesis is the security in microservice 
        architecture when switching from a monolithic architecture to 
        microservices architecture. This thesis presents the key security 
        considerations and presents solutions to the major security aspects.
    \end{sloppypar}
    \begin{sloppypar}
        % (2) aineisto ja menetelmät (erittäin lyhyesti);
        The research is carried out as a literary study.
        The literature consists of articles and textbooks.
        The articles used in this study were gathered from Scopus, 
        GoogleScholar, and internet sources that were deemed reliable.
    \end{sloppypar}
    \begin{sloppypar}
        % (3) tulokset (tälle enemmän painoarvoa); 
        The main security considerations are communication, configuration, 
        and authentication and authorization. 
    \end{sloppypar}
    \begin{sloppypar}
        % (4) johtopäätökset (tälle enemmän painoarvoa).
        Security must be considered as early as possible in a project 
        and the use of tools should be encouraged.
    \end{sloppypar}
\end{enabstract}
