% Avainsanojen lista pitää merkitä main.tex-tiedoston kohtaan \KEYWORDS.

\begin{fiabstract}
    \begin{sloppypar}
        Verkkopalvelut on usein toteutettu käyttäen monoliittista
        arkkitehtuuria. Tämä arkkitehtuuri ei skaalaudu eikä mahdollista
        ketterää kehitystä. Mikropalveluarkkitehtuurin käyttö mahdollistaa nämä
        sekä useita muita etuja.
    \end{sloppypar}
    \begin{sloppypar}    
        % (1) aihe, tavoite ja rajaus 
        Työn aiheena on tietoturva siirryttäessä monoliittisesta
        arkkitehtuurista mikropalveluarkkitehtuuriin verkkopalveluissa. Työn
        tarkoituksena oli selvittää keskeisimmät tietoturvakysymykset
        arkkitehtuurin vaihdoksessa ja esittää löydettyihin ongelmakohtiin
        ratkaisuja. Työ toteutettiin kirjallisuustutkimuksena.
    \end{sloppypar}
    \begin{sloppypar}        
        % (3) tulokset (tälle enemmän painoarvoa); 
        Tietoturvahaasteita ovat mikropalvelujen välinen viestintä, käyttöönotto
        ja konfiguraatio sekä käyttäjän tunnistaminen ja valtuuttaminen.
    \end{sloppypar}
    \begin{sloppypar}
        % (4) johtopäätökset (tälle enemmän painoarvoa).
        Työssä havaittiin, että mikropalveluarkkitehtuuriin siirtymisessä on
        merkittäviä tietoturvariskejä, jotka tulee huolellisesti eritellä ja
        ratkaista jokaisessa arkkitehtuurin vaihdoksessa tapauskohtaisesti.
        Tietoturvan suunnittelu ja toteutus tulee suorittaa suurta
        huolellisuutta noudattaen ja mahdollisuuksien mukaan valmiita
        toteutuksia käyttäen.
    \end{sloppypar}
\end{fiabstract}

\begin{enabstract}
    \begin{sloppypar}
        There are many web services in use that were designed and implemented
        using the monolithic architectural style before the introduction of
        microservices. The monolithic style does not scale or lend itself to
        agile development practices as well as the microservices architectural
        style does. Thus, these services might benefit from an architectural
        transformation to using microservices. Since the architectures are very
        different, the security aspects need to be considered.
    \end{sloppypar}
    \begin{sloppypar}
        % (1) aihe, tavoite ja rajaus 
        The topic of this bachelor's thesis is the security in the microservice
        architecture in the transition from a monolithic architecture to
        microservices. The thesis presents the key security considerations and
        possible solutions to these security issues. The research is carried out
        as a literary study.
    \end{sloppypar}
    \begin{sloppypar}
        % (3) tulokset (tälle enemmän painoarvoa); 
        The main security considerations are inter-microservice communication,
        deployment and configuration, and authentication and authorization.
        Security must be considered as early as possible in a project and the
        use of existing tools and solutions should be encouraged.
    \end{sloppypar}
\end{enabstract}
