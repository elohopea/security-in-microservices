
% --------------------------------------------------------------------


\section{Introduction}
\begin{sloppypar}
    In recent years, mobile applications and the web services which cater to
    them have revolutionized our daily lives by infiltrating social life,
    shopping and almost every aspect of our existence. The rapid expansion and,
    at times, even faster decline of these web services needs a matching
    architecture to meet their very specific needs.
\end{sloppypar}
\begin{sloppypar} One possible solution to address these architectural needs
     is the microservice architecture (MSA). There are multiple definitions for
     MSA but in general the web service is implemented using multiple small,
     independent services that act together.
     \citet{newman2019} definition for a microservice is a service that: is
     independently deployable, is modeled around business domain, that owns the
     data that they need to operate, that communicates via network, is
     technology agnostic, that encapsulates data storage and retrieval and that
     has a stable interface.
\end{sloppypar}
\begin{sloppypar}
    There are many web services already in use which were designed and
    implemented before the onslaught of microservices. Some of these services
    have already made the switch such as Netflix, but this is not the case for
    the whole industry.
\end{sloppypar}
\begin{sloppypar}
    When new development is carried out by a startup, the initial architecture
    might still be a monolith one. Also, when a new service is being created,
    the business domain might not be established yet. Additionally, there might
    exist a fair amount of uncertainty in what exactly is to be developed.
    \citet{newman2019} states that, due to limited software development
    resources, a monolith might be a better fit to the companies that are still
    trying to navigate their way to the product they eventually will offer. In
    the case of success, the need to rapidly scale up the offering emerges.
    \citet{newman2019} refers to these companies as ``scale-ups".
    \citet{newman2019} also states that it is much easier to refactor an
    existing service than to create a new one, and thus the need to split
    monoliths to microservices is and probably will be relevant to the near
    future. This need is further amplified by the agile software development
    methods in which changes in requirements are welcome even in the later
    stages of the development process \citep{agilemanifesto}.
\end{sloppypar}
\begin{sloppypar}
    \citet{10.1007/978-3-319-74433-9_3} find that, as the code base becomes
    large, the monolith architecture (MA) leads to slower development. This is
    due to the complexity inherent in the entwined monolith. As the development
    activity becomes more time consuming, more developers are needed to complete
    any changes to the code base.
\end{sloppypar}
\begin{sloppypar}
    Another reason for converting existing applications to microservices is the
    availability of skilled developers. New developers entering the workforce
    have a very different mindset than the older more seasoned professionals.
    Thus, it is clear that the ways of working and paradigms used are constantly
    changing. The Stack Overflow annual survey \citep{sosurvey2019} conducted on
    developers found that half of the respondents identified as full-stack or
    back-end developers. 40\% of the respondents had less than five years of
    professional experience. Thus, the manpower that would be able and willing
    to keep the old monoliths running is not available.
\end{sloppypar}
\begin{sloppypar}
    Microservices are not the proper choice for all web services
    \citep{newman2019}. However, microservices offer multiple benefits such as
    easier scalability and more modular structure for the application. When the
    architecture needs to be changed, the process needs to happen in an orderly
    and safe way. Also, often initially overlooked security aspects need to be
    identified and addressed as early as possible.
\end{sloppypar}
\begin{sloppypar}
    The microservice architecture (MSA) differs in many ways from the more
    traditional monolithic architecture (MA). When the architecture is changed,
    this shift entails very specific security issues. In this thesis, the MSA and
    related security literature is surveyed and the main differences between MA
    and MSA on security aspects are discussed.
\end{sloppypar}

\begin{sloppypar}
    Implementing security is hard. A theoretical proof for this can be found in
    \citet{andersson2001information}. His main finding was that an attacker has
    an advantage over the developers who are trying to defend a system. For the system
    to be secure, the defending developers must find all the bugs, whereas the
    attacker needs to find only one. Therefore, to make systems more secure, the use
    of existing tools and frameworks that have been tested and are already in use is
    preferable to developing one's own solution for a well-known and already
    solved problem.
\end{sloppypar}
\begin{sloppypar}
    The thesis is organized as follows. The next chapter compares the two
    architectures and the third chapter discusses the changing of architecture
    from MA to MSA. The fourth chapter discusses authentication and
    authorization. The fifth chapter discusses communication. In the seventh
    chapter, other relevant security concerns are discussed. The eigth and
    lastchapter in this thesis presents the conclusions and outlines further
    research topics.
\end{sloppypar}


% --------------------------------------------------------------------


\section{Architectural Comparison}
\begin{figure}[h]
    \includegraphics[scale=0.75]{{monolith_architectur.png}}
    \centering
    \caption{Traditional monolithic architecture
    \citep{10.1007/978-3-319-74433-9_3}}
    \label{fig:monolith}
\end{figure}
\begin{sloppypar}
    MA can be visually presented as in figure \ref{fig:monolith}. The web
    service is a layered structure in which all the different layers have a
    specific task to perform. It follows the Model-View-Controller (MVC) design
    pattern \citep{trygvemvc}. The UI is the View, the business logic is the
    Controller, and the database is the Model.
\end{sloppypar}
\begin{sloppypar}
    The classical definition of MSA is the one given in the
    \citet{fowlerlewisms}. A compatible definition for a single microservice can
    be found in \citet{newman2019}.
\end{sloppypar}
\begin{sloppypar}
    \citet{fowlerlewisms} define the microservice style as follows: the service
    is componentized into smaller services so that each component is
    independently replaceable and upgradeable, the services are organized around
    business capability rather a design pattern (such as the previously
    mentioned MVC), a team should be responsible of their product for its full
    service life, the services are to contain the logic and communicate using a
    communication system without business logic, which can be summarized as
    ``smart endpoints and dumb pipes", decentralized governance meaning that
    choices such as the technology to use or the architecture are not dictated
    to the developers, decentralized data management, infrastructure automation,
    fault tolerance of the whole application since individual services might
    fail or become unavailable, and evolutionary design.
\end{sloppypar}
\begin{sloppypar}
    Newman's (2019) definition for a microservice is a service that it is
    independently deployable, is modeled around business domain, owns the data
    that it needs to operate, communicates via a network, is technology
    agnostic, encapsulates data storage and retrieval, and has a stable
    interface.
\end{sloppypar}
\begin{figure}[h]
    \includegraphics[scale=0.75]{{MSA.png}}
    \centering
    \caption{Microservice Architecture \citep{10.1007/978-3-319-74433-9_3}}
    \label{fig:microservice}
\end{figure}
\begin{sloppypar}
    An example of MSA, presented in figure \ref{fig:microservice}, has many
    problem areas of which one is the challenging security implementation. This
    is because every microservice accessible to the client can also be accessed
    or contacted by other, more malicious parties in the same network. The
    network in the case of web services is the internet. The attack surface
    available for the malicious party is the entirety of the APIs offered by the
    microservices.
\end{sloppypar}
\begin{sloppypar}    
    One solution to limit the attack surface is the addition of API Gateway to
    the architecture as in figure \ref{fig:microserviceapigateway}.
    \citet{DBLP:journals/corr/MontesiW16} present an API Gateway design pattern.
    In this pattern, there exists only one web service accessible to clients.
    The API Gateway is a natural place for a Policy Enforcement Point (PEP) and
    other more MSA-specific features such as service discovery. The security
    features can be implemented in the API Gateway, making it a critical
    component. Since all communication either flows through or is sanctioned by
    the API Gateway, its performance and accessibility are critical.
\end{sloppypar}
\begin{figure}[h]
    \includegraphics[scale=0.75]{{MSA_API_GATEWAY.png}}
    \centering
    \caption{Microservice architecture with API Gateway
    \citep{DBLP:journals/corr/MontesiW16}}
    \label{fig:microserviceapigateway}
\end{figure}

% --------------------------------------------------------------------

\section{Changing the architecture}
\begin{sloppypar}
    The changing of the architecture of an already deployed service from MA to
    MSA should be a gradual process. This ensures a smooth transition and
    minimizes outages to the customers. Sometimes though, this is not possible.
    \citet{newman2019} states that, when a monolithic application is implemented
    following a design patterns such as the MVC \citep{trygvemvc}, this can lead
    to difficulties in the refactoring. The reason is that the code base is not
    split according to the business domain but follows a rigid design pattern.
\end{sloppypar}
\begin{sloppypar}
    For the process to be as simple as possible, the MA is or at least should be
    split into modules with separation of concerns \citep{secchalmsa}. The
    splitting of the monolith can be carried out in various ways, one of which
    is Domain Driven Design \citep{evans2003}. The selection of boundaries for
    the services is critical. If this is done incorrectly, all the affected
    services need to be refactored to mend the error \citep{newman2019}.
\end{sloppypar}
\begin{sloppypar}
    MSA differs from MA in fundamental ways. According to \citet{fowlerlewisms},
    one main difference is the communication between the components. In a
    monolithic application, the processes can send function calls or method
    invocations among themselves. In MSA, the messaging is based on sending
    messages or HTTP requests over the network. 
\end{sloppypar}
\begin{sloppypar}
    Communication using the network is slow, compared to local function calls.
    Function calls entail a stack frame creation in the call stack, execution of
    the function code, and finally popping the stack frame and returning the
    result. Compilers can optimize the code further and inline the function
    calls to eliminate the stack frame creation and the following procedures.
\end{sloppypar}
\begin{sloppypar}
    Requests sent to other microservices through a network are much slower than
    function calls within one computer. Therefore, the communication patterns
    should be changed to consider the change in the communication path. If the
    architecture is changed in such a way that the previous communication model
    among the components is preserved, there would be an excessive amount of
    communication and the resulting system would not be as performant
    \citep{fowlerlewisms}. 
\end{sloppypar}
\begin{sloppypar}
    \citet{webdelays} studied the response times of web sites offered to the
    public. The website response times where measured in seconds. These times
    can be viewed as being at the extreme end. \citet{Johansson_2019} did an
    experiment on a monolithic and a microservices application. The recorded
    response times for the monolithic application where, on average, 64\% faster
    than for the microservice application.
\end{sloppypar}

\begin{sloppypar}
    As has been already established above the communication within the web
    service differs greatly in MA and in MSA. Secure communication is one of the
    key security aspects to be discussed in this paper. In addition to
    communication, the authentication and authorization are another critical
    security concern to be discussed. These will be addressed in the next
    chapter.
\end{sloppypar}

% --------------------------------------------------------------------

\section{Authentication and Authorization}
\begin{sloppypar}
    In the cases where the user must be authenticated, the web service needs a
    way to do this securely. There are many authentication schemes available,
    but users prefer the password \citep{passwordisdead}. Due to this, the
    authentication is usually done with a pair of user credentials that contains
    a username and a password. The user is authenticated, and a key or token is
    transmitted to the user via the network. This communication in both MA and
    MSA should be encrypted in a way that none of the actors in the transfer
    path can intercept the message and misuse the credentials. 
\end{sloppypar}
\begin{sloppypar}
    As has been mentioned previously, implementing security is hard and resource
    intensive. Therefore, the authentication implementation should adhere to an
    already existing framework or another entity providing the authentication.
    This is the case for both the MA and MSA. The available choices include
    Lightweight Directory Access Protocol (LDAP), OpenID Connect (OIDC),
    Security Assertion Markup Language (SAML), and Kerberos.
\end{sloppypar}
\begin{sloppypar}
    Regardless of the authentication scheme, the service or services
    authenticating the user contain the necessary information to do so. Since,
    the services in MSA are to be individually deployable, and a service should
    own its own data, implementing the authorization as part of the API Gateway,
    as a single micro service or using a third party authentication would be
    beneficial. The information would not be spread in to multiple services and
    multiple services would not have to have related implementation. In
    addition, whenever information is available, it is a target for thieves and
    hackers. Regardless of the implementation choices, the service must adhere
    to practices that minimize the risks of data breaches. 
\end{sloppypar}

\subsection{Authentication and Authorization in MA}
\begin{sloppypar}
    In MA, it is possible to implement features in such ways that a process in
    which the application runs has access to a session or a user object that
    carries the user information. This information can consist of the granted
    roles and rights for the user. The information can be queried easily and
    securely when access control is needed.
\end{sloppypar}
\begin{sloppypar}
    An example of a web service implemented in MA is presented in figure
    \ref{fig:MA_AUTH}. The user sends a request to a load balancer, which has
    all the information on the currently operational services. All the services
    run in the same process in which an authentication and authentication
    service or functionality is present. When a user first accesses the service,
    credentials are verified and, in a successful case, a session or a user
    object is created for the process.
\end{sloppypar}
\begin{figure}[h]
    \includegraphics[scale=0.70]{{MA_AUTH.png}}
    \centering
    \caption{Traditional load balanced MA web service \citep{authinmsa}}
    \label{fig:MA_AUTH}
\end{figure}

\subsection{Authentication and Authorization in MSA}
\begin{sloppypar}
    Authorization of the user rights can be implemented in various ways. One of
    them is an authorization service which can contain the access control list.
    Services verify from the authorization service that the client user or the
    role is allowed to access the requested service or functionality. The
    authorization is verified in memory and without any communication over the
    network.
\end{sloppypar}
\begin{figure}[h]
    \includegraphics[scale=0.75]{{MSA_AUTH_SERVICE.png}}
    \centering
    \caption{MSA authorization service \citep{authinmsa}}
    \label{fig:MSA_AUTH_SERVICE}
\end{figure}
\begin{sloppypar}
    In MSA, accessing the access control lists is not as easy as it is in MA. In
    order to verify that a specific right exists, the service would have
    communication with the authorization service. This communication would need
    to happen every time a user tries to access a functionality with access
    restrictions. This could potentially lead to excessive communication from
    all the services and form a performance bottleneck at the authorization
    service. An example of such architecture is presented in figure
    \ref{fig:MSA_AUTH_SERVICE}.
\end{sloppypar}
\begin{figure}[h]
    \includegraphics[scale=0.75]{{TOKEN_AUTH.png}}
    \centering
    \caption{Bearer token-based authentication \citep{authinmsa}}
    \label{fig:TOKEN_AUTH}
\end{figure}
\begin{sloppypar}
    To limit the excessive communication, a token-based authentication and
    authorization scheme can be used. A simplified process flow is presented in
    \ref{fig:TOKEN_AUTH}. A user enters credentials to a login, and with these
    credentials, an authentication and authorization service grants the user a
    token. This token is used to access the services in the system. The services
    in question trust the issuer of the token and verify the token. If the token
    is accepted and access can be granted the request is serviced without any
    communication between the service and the authorization service. A specific
    token-based authentication protocol is discussed in the next chapter.
\end{sloppypar}

\subsection{OIDC}
\begin{sloppypar}
    OIDC is an identity layer to accompany the OAuth 2.0 authorization
    framework-based protocol. 
\end{sloppypar}
\begin{figure}[h]
    \includegraphics[scale=0.75]{{OIDC_FLOW.png}}
    \centering
    \caption{OIDC steps \citep{oidcflow}}
    \label{fig:oidcflow}
\end{figure}
\begin{sloppypar}
    An example of an OIDC flow is presented in figure \ref{fig:oidcflow}. The
    process flow steps are the following: the client application requests an
    authentication from the identity provider, the end user is authenticated and
    authorization is obtained, the identity provider responds to the initial
    authentication request by sending an ID token and a possible access token,
    the client application can request the end user claims from the
    authorization server with the access token, and finally the authorization
    service returns the claims to the client application. The presented flow, is
    like the OAuth 2.0 flow, and the token sent to the client application is a
    JavaScript Object Notation Web Token (JWT).
\end{sloppypar}

\subsection{JWT}
\begin{sloppypar}
    JWT is a format to represent claims. It consists of base64 encoded, point
    separated strings, which can easily be carried in the HTTP request or
    response. The contents are key value pairs, and the token may or may not be
    signed and encrypted \citep{RFC7519}. The token may contain an expiration
    time. If the token is used to validate requests without a server-side
    implementation that can revoke a token, it will be valid until the
    expiration time. The JWT token must be issued by an authority trusted by the
    service.
\end{sloppypar}
\begin{sloppypar}
    The signing of JWT can be carried out in various ways. These are presented
    by \citet{RFC7515}. The signature is computed with the algorithm and keys or
    certificates specified in the header values. When the token is signed with a
    private key, it can be verified by all parties in possession of the
    corresponding public key.
\end{sloppypar}

\subsection{Opaque Token}
\begin{sloppypar}
    \citet{authinmsa} compared several authentication and authorization
    solutions. One of the solutions discussed in the paper was the usage of an
    opaque access token for the client to API-Gateway communication, which is
    mapped to JWT in the API Gateway. The JWT would be used in all communication
    as the means to authenticate and authorize the user. The main problem the
    opaque token is said to solve is the logout problem. In comparison, a bearer
    token is valid until its expiration, and the client cannot invalidate the
    token. The opaque token flow is presented in figure
    \ref{fig:OPAQUE_TOKEN_API_GATEWAY}.
\end{sloppypar}
\begin{sloppypar}
    The use of two tokens has many advantages. The token granted to the client
    allows access only to the API Gateway. This token cannot be used to access
    the services directly. Furthermore, the opaque token be can revoked by
    simply removing it from the storage. The internal JWT can alleviate some of
    the issues that arise when MA is to be changed to MSA. Firstly, token theft,
    though still possible, is limited to the time the client is logged on to the
    system. Secondly, though not advisable, the JWT can carry session
    information and the whole session. This allows for more of the previous MA
    implementation to be used without as many changes.
\end{sloppypar}
\begin{figure}[h]
    \includegraphics[scale=0.75]{{OPAQUE_TOKEN_API_GATEWAY.png}}
    \centering
    \caption{API Gateway and opaque token \citep{authinmsa}}
    \label{fig:OPAQUE_TOKEN_API_GATEWAY}
\end{figure}

% --------------------------------------------------------------------

\section{Communication}
\begin{sloppypar}
    As already explained, the service components in MA can communicate using
    events, procedure calls or other methods available within a single server
    machine. Usually all this communication stays within a single computer and
    thus does not easily compromise confidentiality. This is not the case in
    MSA.
\end{sloppypar}
\begin{sloppypar}
    In MSA, the services communicate with each other via a network. There are
    multiple protocols and messaging systems to choose from such as, REST API,
    Advanced Message Queuing Protocol (AMQP), Enterprise Service Bus (ESB), and
    Remote Procedure Calls (RPC). For a more complete list view
    \citet{secchalmsa}. In this paper REST API will be examined further.
\end{sloppypar}

\subsection{Representational State Transfer (REST)}
\begin{sloppypar}
    \citet{restroy} presented REST in 2000. REST has become a very successful
    architectural style. The style was defined using various constraints, one of
    which is stateless communication. This entails that a request must contain
    all the information needed to fulfill the request because the server does
    not keep track of the client. All session state is stored in the client, of
    which the server has no prior knowledge before receiving the request.
    \citet{secchalmsa} critique the REST paradigm from the security perspective.
    It states that the design of the architecture does not meet the security
    requirements for web applications. She also states that REST does not allow
    for any server-side sessions and thus token revocation is impossible. Tokens
    can be validated only for the correct issuer by signature and for
    expiration. As such tokens, are more compatible with REST, but there still
    must be the public keys in the server for the signature verification.
\end{sloppypar}

\subsection{Coping With Failure in Communication}
\begin{sloppypar}
    \citet{DBLP:journals/corr/MontesiW16} present widely used design pattern for
    MSA. The Circuit Breaker can be used to mitigate the case where a
    microservice operates slower than the services calling it and is unable to
    fulfill the requests. The circuit breaker is either implemented in the
    microservice or as a proxy between the client service and the microservice.
    When the microservice does not service requests as intended, the circuit
    breaker trips and sends a failure message to the clients immediately when
    requests are received, thus allowing the overloaded microservice time to
    service the prior requests.
\end{sloppypar}
\begin{sloppypar}
    The circuit breakers can prevent an application from becoming completely
    unresponsive and crashing when a denial of service attack is carried out on
    the service.
\end{sloppypar}

% --------------------------------------------------------------------

\section{Deployment Automation and Production}
\begin{sloppypar}
    Both MA and MSA web services can be installed on servers operated by the
    organization or individuals themselves. The software can be installed on the
    host operating system directly or a virtualization technology can be used.
\end{sloppypar}
\begin{sloppypar}
    \citet{closer18} implemented a test system mimicking the Deutsche Bahn seat
    reservation system using MSA. The purpose of the test system was to analyze
    security risks that were introduced by the implementation. In the study,
    they categorized the solution into three layers, the first of which was the
    compute provider, the second was the encapsulation technology, and the third
    one was the deployment. The technologies for these layers were: Amazon Web
    services, Docker for containers, and Kubernetes (k8s) nodes, respectively.
    They found out that the cloud-based infrastructure, when used in MSA,
    resulted in a more complex solution than in MA. The added layers such as the
    K8s, must be configured correctly, and an error in one could potentially
    compromise the whole system. In addition, the implementation of security is
    difficult and resource intensive. The rewards from good security, on the
    other hand are invisible. When microservices are implemented or even
    planned, the security should be considered as early as possible.
    Implementing security later in the project or as an afterthought can be more
    expensive and difficult.
\end{sloppypar}

\subsection{Virtualization}
\begin{sloppypar}
    Virtualization can be carried out in various ways but, in this thesis, it
    means virtual machines (VM) and containerization. VM is a complete
    installation of all the software needed for a system to run. In its basic
    form, a container uses the host operating system capabilities without the
    need to install an operating system or nonessential software anew. Only the
    application and its dependencies are needed. VMs are run on the host by a
    hypervisor and containers by an engine or the host operating system. Running
    an application on a VM or in a container does not differ for the application
    or, in this case, for the web service, as the environment is similar.
\end{sloppypar}
\begin{sloppypar}
    Containers have considerably lower overhead when compared to VMs. A
    container can be created and started easily and automatically as needed by
    an orchestration solution such as Docker Compose, Docker Swarm or k8s. All
    these tools must be correctly configured and used according to their
    specification and best practices.
\end{sloppypar}

\subsection{Orchestration}
\begin{figure}[h]
    \includegraphics[scale=0.75]{{orchestration.png}}
    \centering
    \caption{Orchestration constituents \citep{containernetworking}}
    \label{fig:orchestration_constituents}
\end{figure}
\begin{sloppypar}
    Container orchestration can consist of the services and operations as is
    depicted in \ref{fig:orchestration_constituents}. Scaling refers to
    automatically creating or shutting down pods or containers to match the
    utilization level. The containers and the microservices within can have a
    new version that needs to be rolled into the production. The upgrade service
    is responsible in doing this. Service discovery is a service which is used
    to locate running services and as a service to which self-registration is
    carried out. If a service becomes nonoperational, it might not be able to
    send the orchestrator any message on this erroneous behavior. Therefore, the
    orchestrator can have a health check service. This service can periodically
    send a message to a service and, if no response is not received within a
    reasonable time, the service is deemed nonoperational and a new one is
    created by the scheduler. This service creates the individual pods or
    containers according to the system settings. The last orchestration part is
    the organizational primitives. These are used to e.g. label pods or
    containers with matching business domain names. This is to make the
    administration and setup work easier \citep{containernetworking}.
\end{sloppypar}

\subsection{Service Mesh}
\begin{figure}[h]
    \includegraphics[scale=0.75]{{service_mesh.png}}
    \centering
    \caption{Service mesh basic architecture \citep{servicemesh}}
    \label{fig:service_mesh_basic_architecture}
\end{figure}
\begin{sloppypar}
    A service mesh tries to solve the service-to-service communication
    challenges and to allow for monitoring of the entire system. In figure
    \ref{fig:service_mesh_basic_architecture}, a very basic service mesh
    architecture is presented. The architecture consists of a Control plane and
    a Data plane. The Control plane offers a user interface for system
    administrators, a Policy Information Point (PIP) and a Policy Decision Point
    (PDP), and collects metrics of the behavior and actions in the
    system. The data moves in the dataplane. Each microservice is accompanied by
    a proxy. All requests and responses flow through the proxies and are
    controlled by the Control plane \citep{servicemesh}. The microservices do
    not have to be aware of the proxy that is in the information path.
\end{sloppypar}
\begin{sloppypar}
    There are several products available for a service mesh, such as Istio,
    Linkerd, and Consul Connect, to mention a few. As was the case with
    authentication and security in general, it is not advisable to implement
    security-critical features if there exists an off-the-shelf alternative.
\end{sloppypar}
\begin{sloppypar}
    The use of a service mesh greatly simplifies the microservice implementation
    since the proxy can contain many of the features that would otherwise be
    repeated in all the services. The service mesh can provide a certificate
    authority for the communication between the services. The proxies act on
    behalf of the service, and all proxy-to-proxy communication can be encrypted
    on the transport layer with TLS \citep{pathtoservicemesh}. Furthermore, it
    facilitates the use of mTLS where both parties are verified. A service mesh
    also allows for multi-cloud installations. 
\end{sloppypar}

% --------------------------------------------------------------------

\section{Other Security Concerns}
\subsection{Known Vulnerabilities in JWT}
\begin{sloppypar}
    The choices for the algorithm for signing the JWT includes "none". This was
    found to be troublesome by \citet{nonejwt}. He found that many libraries did
    not operate in the desired way. The receiving party could be fooled to
    validate a mutated token without any signature with the "none" as its
    algorithm. In addition to this vulnerability, \citet{nonejwt} found that the
    verification suffered from another fatal flaw. A server can be fooled into
    to accepting invalid tokens. A malicious party can sign a JWT with the
    server public key and a symmetric algorithm. The token verification
    functions in some of the JWT libraries do not have the algorithm as one of
    the parameters and allow the algorithm field in the token to dictate the
    verification algorithm. If the keys match, as they should if the public key
    was correct, the library accepts the forged token as valid.
\end{sloppypar}

\subsection{Transactions}
\begin{sloppypar}
    The data in the system is the resource to be protected. If this data becomes
    corrupted, the system is not secure. In some cases, the actions to be taken
    consists of multiple reads and writes of the data, and the order of
    execution changes the result.
\end{sloppypar}
\begin{sloppypar}
    Transactions can be used when updating database content to make sure that
    atomicity, consistency, isolation, and durability (ACID) \citep{acid} are
    followed. When using MSA, according to the definition, each of the
    microservices should contain or have access to its own data i.e. a database.
    In MA, transactions are easier to implement, especially when the execution
    is done sequentially. In MSA, this is not necessarily the case. Performing
    an action might entail calling various microservices and if any of the
    individual actions fail permanently the changes that have already been made
    by other actions need to be undone. In addition to this, HTTP is
    asynchronous, and the execution order cannot be guaranteed.
\end{sloppypar}
\begin{sloppypar}
    Transaction-related issues can be solved by either splitting the monolith
    in such a way that the actions that need to be carried out as transactions
    are carried out in one microservice, or by creating a service to
    coordinate the actions taken by the single microservices.
\end{sloppypar}

\subsection{Software Development}
\begin{sloppypar}
    When software is developed using MA, it is usually deployed as a whole and
    the program code can be compiled, tested, and used as a single unit or
    multiple modules. In contrast, a service implemented in MSA can be deployed
    in single-microservice units, and thus each component can be worked upon
    individually and deployed once ready. 
\end{sloppypar}
\begin{sloppypar}
    The immediacy in the deployment of the microservices entails a very specific
    security risk. \citet{integinside} present threats from malicious insiders
    working on the services as developers or in other positions with access to
    sensitive information. In microservice development, the finished
    implementations are immediately released to production. There are steps in
    the CD pipeline prior to this, but once tests pass in the test environments,
    the pipeline is supposed to publish the changes to the actual production
    environment. The paper presents four specific threats. The first one is that
    the knowledge of sensitive information is spread among the developers more
    widely than in MA. This is due to access needs by developers. The second
    threat is that the insiders monitoring and operating the running system
    intentionally harm the system by making malicious changes. The third threat
    is the developers know the configurations and have the ability to make
    almost instantaneous changes to them or to the microservices themselves. The
    last presented threat in the paper is non-repudiation. The system is not
    able to disallow malicious requests when the developers have had access to
    the keys and other configurations. They can effectively implement services
    that emit malicious requests or responses. Malicious modification attempts
    in an MA are more easily screened by performing security audits and by peer
    reviewing the code. In an MSA, the knowledge of a single service and its
    inner workings are shared by a more limited number of people. Finding the
    compromised actions from the interoperation of the distinct microservices is
    a daunting task.
\end{sloppypar}

\subsection{Externalized Configuration}
\begin{sloppypar}
    To allow for easy configuration change management, there should exist a
    configuration orchestration service. This service should have an API from
    which services during their startup can load their appropriate configuration.
    The configuration of the whole system can be easily maintained through the
    API.
\end{sloppypar}
\begin{sloppypar}
    The contents of the configuration are highly sensitive information. It can
    consist of addresses, secrets, and other information that alter the behavior
    of the system. Secrets refer to credentials, connection strings, other keys
    and similar items that are to be kept confidential. Therefore, the content
    must be stored safely and not allowed to be read or altered by unauthorized
    users.
\end{sloppypar}

\subsection{Logging}
\begin{sloppypar}
    Logging in MA is relatively easy. The chosen logging solution is used, and
    logs are created in easily configurable locations. In MSA, this can be more
    difficult. Each of the microservices needs to have its own logger, and each
    of these needs to be configured to log to a proper location. The logs that
    are created need to be persisted for the length of time allowed by
    legislation and according to the need of the system administration.
\end{sloppypar}
\begin{sloppypar}
    Without logs, it is impossible to verify correct operation of the system, nor
    is it possible to gain knowledge of a possible security breach. 
\end{sloppypar}

\subsection{Defense-in-Depth (DiD)}
\begin{sloppypar}
    It is not enough to secure the boundary between the perimeter and the
    internal system. In DiD, the system should implement security measures at
    multiple layers within the system \citep{iec62443}. The system can be
    presented layered as in figure \ref{fig:Defense_in_depth}. Each of these
    layers should have security features, and a breach in one layer should not
    compromise the features of an inner layer in the system.
\end{sloppypar}
\begin{sloppypar}
    \citet{moderndid} takes the idea of layered defense even further and
    proposes an integrated approach to DiD. He defines the defense layers as
    edge routers, DDos defenses, Managed DNS, Reverse proxies, Bot Management,
    Web application firewalls, API defenses, and Caching. He further suggests
    that these layers or lines of defense should be aware of each other and be
    accessible from a single UI. Another approach he defines as human expertise.
    In this approach, there exists a command center that is always manned and,
    as such, able to respond to encountered threats. In some instances, this
    might be applicable, but for all web services it is not feasible.
\end{sloppypar}
\begin{figure}[h]
    \includegraphics[scale=0.75]{{Defense_in_depth.png}}
    \centering
    \caption{Defense-in-Depth}
    \label{fig:Defense_in_depth}
\end{figure}

% --------------------------------------------------------------------

\section{Conclusion}
\begin{sloppypar}
    This paper discussed the security aspects of changing the architecture from
    MA to MSA.
\end{sloppypar}
\begin{sloppypar}
    In practice, the previous MA that has been changed to MSA can have more
    aspects to go wrong than in the previous MA implementation. The deployment
    can entail installing, virtualization, monitoring, and other tools. In some
    cases, these tools must be implemented by in-house developers and, thus,
    more costs are incurred upfront and in the upkeep of the system. In addition
    to being more costly, the development has higher security risks involved.
\end{sloppypar}
\begin{sloppypar}
    MSA has higher complexity due to more tools needed and having more
    potentially exposed attack surface. Security can be thought of as being only
    as good as its weakest link. In general, an MSA deployment has multiple
    layers, which all must be consistent and correct. One example is the
    configuration of the operating system on the server running the
    virtualization environment. All the layers from the server hardware to the
    handling of errors in the application code must be of high quality to
    prevent failures in security.
\end{sloppypar}
\begin{sloppypar}
    The communication that was in monolith a simple in-process call might not be
    possible as such in an MSA web service. The individual services communicate
    via the network with high overhead, in comparison to a simple function call.
    Furthermore, the identity and authorization of the entity requesting an
    action or data can usually be trusted in an in-process call. The mechanisms
    for proper authentication and authorization amount to even higher overhead
    for the MSA. There exists a very real risk for the development team to
    implement an insufficient security scheme. \citet{maninthemachine} outline
    and implement a Man-in-the-machine attacks in which an insecure
    communication even within a single computer can be compromised. All network
    communication should be between verified counterparties and should use
    secure measures, even when this counterparty is running on the same
    server.
\end{sloppypar}
\begin{sloppypar}
    In MSA the security should be implemented in depth. There must be a healthy
    mistrust on all requests, and security should be built into the system.
    Security must be considered right from the beginning of the project in which
    the application architecture is transformed. The choices made in the
    development of the web service when following the MA do not carry to the MSA
    without changes.
\end{sloppypar}
\begin{sloppypar}
    The use of pre-existing tools and frameworks is highly encouraged. The tools
    currently available such as Docker, k8s, and Istio solve many of the
    inherent security issues if applied correctly. The tools and frameworks that
    exist cannot be blindly trusted. As an example, some of the implementations
    using JWT had and in some cases still have serious flaws and are not secure.
\end{sloppypar}
\begin{sloppypar}
    Authentication and authorization in MA and in MSA can differ greatly. In MA,
    a session-based authentication and authorization is applicable. This is not
    the case in MSA, especially if a REST API is used. In REST API, all the
    information needed to serve a request should be enclosed in the request.
    Tokens such as JWT cater to this and can carry user information and other
    claims. The tokens are issued and signed by a trusted party. An API Gateway
    can act as a PEP and allow or disallow a request. Also, if a service mesh is
    used such as Istio, the enforcement can be done in the proxy for a specific
    microservice.
\end{sloppypar}
\begin{sloppypar}
    The results found in this study can be used to determine if an
    architecture change is feasible for a specific application.
\end{sloppypar}
\begin{sloppypar}
    In this thesis, many security aspects were not discussed. How do the many
    available design patterns for the splitting of the monolith fair when
    compared on security aspects? On the communication, only REST API on HTTP was
    discussed in any real extent.
\end{sloppypar}
\begin{sloppypar}
    Further research is needed on the security aspects of the different design
    patterns that are available for the architectural change. Also, the
    implementation of a field level access control should be studied further.
\end{sloppypar}

% Loppuluku päättää työn. Luvun nimi on tyypillisesti ``yhteenveto'' tai
% ``johtopäätöksiä''. Valitse se otsikko, joka tuntuu sopivammalta työsi
% luonteeseen. Joka tapauksessa loppuluku sisältää niin työn yhteenvedon kuin
% johtopäätöksiä työn tulosten perusteella. Pääajatus on antaa lukijalle selvä
% kuva siitä, miten johdannossa asetettuihin tavoitteisiin työssä vastattiin.

% Käsittele loppupuvussa seuraavia asioita (jotakuinkin tässä järjestyksessä):
% %
% \begin{itemize} \item Muistutus työn tavoitteista (sidoksisuus johdantoon)
%   \item Päätulokset kootaan yhteen, pohditaan niiden merkitystä \item
%   Suositukset konkreettisiksi toimenpiteiksi (``Mitä sitten?'' Nyt kun
%   käytössä on tämän työn myötä tullut tieto, mitä se nyt tarkoittaa tälle
%   asialle/alalle.) \item Tulosten soveltuvuus, käyttöön liittyvät rajoitukset
%   \item Jatkotutkimustarve (``Tulevaisuudessa olisi mielenkiintoista
%   selvittää...'' tms.) \item Työn onnistumisen arviointi (Huom! Älä arvioi
%   omaa kirjoitusprosessiasi vaan tekemääsi tutkimusta) \end{itemize}


% --Junk starts here--------------------------------------------------
% --------------------------------------------------------------------
%     \begin{sloppypar} authorization, access control, policy decision point (PDP),
%         policy enforcement point (PEP), A Framework for Policy-based Admission
%         Control https://tools.ietf.org/html/rfc2753 \end{sloppypar}
% --------------------------------------------------------------------
% --------------------------------------------------------------------
%     \begin{sloppypar} \end{sloppypar} \begin{sloppypar}
%     https://www.nginx.com/blog/service-discovery-in-a-microservices-architecture/
%     https://www.consul.io \end{sloppypar}
% --------------------------------------------------------------------
% --------------------------------------------------------------------
%     \section{Orchestration solutions} \begin{sloppypar} Docker Swarm, Kubernetes
%     (K8s), Azure, sandbox, virtualization \end{sloppypar}
% --------------------------------------------------------------------
% --------------------------------------------------------------------
%     \section{Authorization} \begin{sloppypar}
%     https://techbeacon.com/security/microservices-apps-do-identity-access-management-without-overhead
%     \end{sloppypar} \begin{sloppypar} token (JWT, OAuth 2.0), (API gateway, IdP)
%     vs distribution \end{sloppypar} \begin{sloppypar} Identity and Access
%     Management (IAM), Identity federation, Azure Active Directory, Active
%     Directory Federation Services (ADFS), Google, Facebook, Twitter, LinkedIn
%     etc., Single-Sign-On (SSO), Security Assertion Markup Language (SAML), SAML
%     identity provider, OpenId provider, OAuth 2.0, OpenId Connect (OIDC), Identity
%     Management: Identification, Authentication, Authorization. \dots
%     \end{sloppypar}
% --------------------------------------------------------------------
% --------------------------------------------------------------------
%         TODO
%         Next messaging systems list is from \citet{secchalmsa}:
%     lightweight REST API, Sync RPC, GraphQl Async REST, gRPC Apache Kafka, ZeroMQ
%         Java Message Service: ActiveMQ, JBOSS messaging, Glassfish AMQP: RabbitMQ,
%         Qpid, HornetQ MuleESB, Apache ServiceMix, JBossESB heavyweight WebSocket
% --------------------------------------------------------------------
% --------------------------------------------------------------------
%     \subsection{Event-Driven Communication} \begin{sloppypar} \end{sloppypar}
% --------------------------------------------------------------------
% --------------------------------------------------------------------
%     \section{Decision Flowcharts} \begin{sloppypar} In this chapter flowcharts are
%     presented that capture the main findings and the factors affecting decision
%     making in and architectural switch. \end{sloppypar} \begin{sloppypar} In
%     figure \ref{fig:auth_flowchart} a decision flowchart for authentication and
%     authorization is presented. \end{sloppypar} \begin{figure}[h]
%     \includegraphics[scale=0.75]{{AUTH_FLOWCHART.png}} \centering
%     \caption{Authorization flowchart.} \label{fig:auth_flowchart} \end{figure}
% --------------------------------------------------------------------
% --------------------------------------------------------------------
%     The amount
%     of code to refactor is much larger than in a small microservice.
%     Microservice should do one thing and, as such, it should be more
%     understandable.
% --------------------------------------------------------------------
% --------------------------------------------------------------------
% \section{Session} \begin{sloppypar} Client handled session is easy to tamper
% with and has risks involved. The service would either have to trust the client
% offered session or verify a signature. In both of these cases the client
% application would have to be aware of the service and would be able to
% communicate with it directly. This would bring considerable overhead on all
% the services and a security risk. \end{sloppypar} \begin{sloppypar} Sessions
% can be used in MSA when the architecture has an API Gateway. The session is
% stored in the Gateway and a session key is carried in the requests made by the
% client application. The API Gateway then verifies the request and can grant
% access to specific service. Without the Gateway the session would have to be
% either centrally maintained at a session service or be handled by the client
% and passed along in the requests. \end{sloppypar}
% --------------------------------------------------------------------
% --------------------------------------------------------------------
% \subsubsection{Scaling} \begin{sloppypar} \end{sloppypar}
% \subsubsection{Upgrading} \begin{sloppypar} \end{sloppypar}
% \subsubsection{Service Discovery} \begin{sloppypar} Service discovery as
% presented in \citet{DBLP:journals/corr/MontesiW16} is a design pattern in
% which a registry is kept on currently running microservices. The microservices
% register them selves to the service discovery registry. This registry is used
% by either a router to route client service calls to running microservices or
% by the client directly. \end{sloppypar}
% --------------------------------------------------------------------
% --------------------------------------------------------------------
% \subsection{Testing} \begin{sloppypar} The possible ways to fail for a web
% service consisting of a large number of microservices are numerous. One
% solution for testing for outages is the Netflix created Chaos Monkey
% \citep{chaosmonkey}. It is used for resilience testing and can terminate
% containers or VMs at random. \end{sloppypar}
