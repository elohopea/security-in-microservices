
% --------------------------------------------------------------------

\section{Introduction}
\begin{sloppypar}
    In recent years, mobile applications and web services which cater to them
    have revolutionized our daily lives by infiltrating social life, shopping
    and almost every aspect of our existence. The rapid expansion and, at times,
    even faster decline of these web services needs a matching architecture to
    meet their very specific needs.
\end{sloppypar}
\begin{sloppypar}
    There are many web services already in use which were designed and
    implemented before the onslaught of microservices. Some of these services
    have already made the switch such as Netflix but this is not the case for
    the whole industry.  
\end{sloppypar}
\begin{sloppypar}
    When new development is carried out by a startup, the initial architecture
    might still be a monolith one. Also, when a new service is being created the
    business domain might not be established yet. Additionally there might exist
    a fair amount of uncertainty in what exactly is to be developed.
    \citet{newman2019} states that, due to limited resources, a monolith might
    be a better fit to these companies trying to navigate to the actual product
    they are to offer. In the case of success, the need to rapidly scale the
    offering emerges. \citet{newman2019} refers to these companies as
    "scale-ups". \citet{newman2019} also states that it is much easier to
    refactor an existing service than to create a new one and thus the need to
    split monoliths to microservices is and probably will be relevant to the
    near future. This is furthermore amplified by the use of agile software
    development methods in which the change in requirements is welcome even in
    later stages of the development \citep{agilemanifesto}.
\end{sloppypar}
\begin{sloppypar}
    \citet{10.1007/978-3-319-74433-9_3} finds that as the code base becomes
    large the Monolith Architecture (MA) leads to slower development. This is
    due to the complexity inherent in the entwined monolith. As the development
    activity becomes more time consuming, more developers are needed to complete
    the needed and often mandatory changes to the code base.
\end{sloppypar}
\begin{sloppypar}
    New developers entering the workforce have a very different mindset than the
    older more seasoned professionals. Thus, it is very clear that the ways of
    working and paradigms to be used are constantly changing. 
\end{sloppypar}
\begin{sloppypar}
    The Stack Overflow annual survey \citep{sosurvey2019} conducted on
    developers found that half of the respondents identified as full-stack or
    back end developers. 40\% of the respondents had less than five years of
    professional experience. Thus, the manpower that would be able and willing
    to keep the old monoliths running is not available.
\end{sloppypar}
\begin{sloppypar}
    Microservices are not the proper choice for all web services
    \citep{newman2019}. However, microservices offer multiple benefits such as
    easier scalability and more modular structure for the application. When the
    architecture needs to be changed, the process needs to happen in an orderly
    and safe way. Also, often initially overlooked security aspects need to be
    identified and addressed as early as possible.
\end{sloppypar}
\begin{sloppypar}
    Microservice Architecture (MSA) differs in many ways from the more
    traditional Monolithic Architecture (MA). When the architecture is changed
    this shift entails very specific security issues.
\end{sloppypar}
\begin{sloppypar}
    In this thesis, the MSA and related security literature is surveyed and the
    main differences between MA and MSA on security aspects are discussed.
\end{sloppypar}
\begin{sloppypar}
    This thesis is organized as follows. The first chapter of this thesis
    compares the two architectures and the second chapter discusses the changing
    of architecture from MA to MSA. The third chapter presents the main security
    aspects of the change. The fourth and fifth chapter discuss the key security
    aspects that need to be addressed when the architecture is to be changed:
    access control, and system communication, respectively. In the sixth chapter
    other relevant security concerns are discussed. The last chapter in this
    thesis presents the conclusions and further research to be carried out on
    the subject.
\end{sloppypar}


% --------------------------------------------------------------------


\section{Architectural Comparison}
\begin{figure}[h]
    \includegraphics[scale=0.75]{{monolith_architectur.png}}
    \centering
    \caption{Traditional Monolithic Architecture \citep{10.1007/978-3-319-74433-9_3}}
    \label{fig:monolith}
\end{figure}
\begin{sloppypar}
    MA can be visually presented as in figure \ref{fig:monolith}. The web
    service is a layered structure in which all of the different layers have a
    specific task to perform. This follows the Model-View-Controller (MVC)
    design pattern \citep{trygvemvc}. The UI is the View, the business logic is
    the Controller, and the database is the Model.
\end{sloppypar}
\begin{sloppypar}
    For an MSA the classical definition is the one given in the
    \citet{fowlerlewisms}. A compatible definition for a single microservice can
    be found in \citet{newman2019}.
\end{sloppypar}
\begin{sloppypar}
    \citet{fowlerlewisms} define the microservice style as follows: the service
    is to be componentized using services in which a component is independently
    replaceable and upgradeable, the services are organized around business
    capability rather a design pattern such as the previously mentioned MVC, a
    team should be responsible of their product for it's full service life, the
    services are to contain the logic and communicate using a communication
    system without business logic which can be simplified as "smart endpoints
    and dumb pipes", decentralized governance meaning that choices such as the
    technology to use or architecture are not dictated to developers,
    decentralized data management, infrastructure automation, the whole
    application needs to be fault tolerant since individual services might fail
    or become unavailable, and evolutionary design.
\end{sloppypar}
\begin{sloppypar}
    \citet{newman2019} definition for a microservice is a service that: is
    independently deployable, is modeled around business domain, that owns the
    data that they need to operate, that communicates via network, is technology
    agnostic, that encapsulates data storage and retrieval and that has a stable
    interface.
\end{sloppypar}
\begin{figure}[h]
    \includegraphics[scale=0.75]{{MSA.png}}
    \centering
    \caption{Microservice Architecture \citep{10.1007/978-3-319-74433-9_3}}
    \label{fig:microservice}
\end{figure}
\begin{sloppypar}
    An example for MSA, presented in figure \ref{fig:microservice}, has many
    problem areas of which one is the challenging security implementation. This
    is due to the fact that every microservice accessible to the client can also
    be accessed or contacted by other more malicious parties in the same
    network. The network in the case of web services is the internet. The attack
    surface available for the malicious party is the entirety of the APIs
    offered by the microservices.
\end{sloppypar}
\begin{sloppypar}    
    One solution to limit the attack surface is the addition of API Gateway to
    the architecture as in figure \ref{fig:microserviceapigateway}.
    \citet{DBLP:journals/corr/MontesiW16} present an API Gateway design pattern.
    In this pattern, there exists only one web service accessible to clients.
    The API Gateway allows for a natural place for a Policy Enforcement Point
    (PEP) and other more MSA specific features such as service discovery. The
    security features can be implemented in the API Gateway making it a critical
    component. Since all communication is to either flow through or be
    sanctioned by the API Gateway the performance and accessibility are
    critical.
\end{sloppypar}
\begin{figure}[h]
    \includegraphics[scale=0.75]{{MSA_API_GATEWAY.png}}
    \centering
    \caption{Microservice Architecture With API Gateway \citep{DBLP:journals/corr/MontesiW16}}
    \label{fig:microserviceapigateway}
\end{figure}


% --------------------------------------------------------------------


\section{Changing the architecture}
\begin{sloppypar}
    The changing of the architecture of an already deployed service from MA to
    MSA should be a gradual process. This ensures a smooth transition and
    minimizes outages to the customers. Sometimes, though this is not possible.
    \citet{newman2019} states that when a monolithic application is implemented
    following a design patterns such as the MVC \citep{trygvemvc} this can lead
    to difficulties in the refactoring. Since the code base is not split
    according to the business domain but follows a rigid design pattern.
\end{sloppypar}
\begin{sloppypar}
    In order for the process to be as simple as possible, the MA is or at least
    should be split into modules with separation of concerns \citep{secchalmsa}.
    The actual splitting of the monolith can be carried out in various ways, one
    of which is Domain Driven Design \citep{evans2003}. The selection of
    boundaries for the services is critical. If this is done incorrectly all the
    affected services need to be refactored to mend the error
    \citep{newman2019}.
\end{sloppypar}
\begin{sloppypar}
    The MSA differs from a MA in fundamental ways. According to
    \citet{fowlerlewisms}, one main difference is the communication between the
    components. In a monolith application the processes can send function calls
    or method invocations among themselves. Where as in MSA, the messaging is
    based on sending messages or HTTP requests. 
\end{sloppypar}
\begin{sloppypar}
    Communication using the network is extremely slow compared to local function
    calls. Function calls entail a stackframe creation in the call stack,
    execution of the function code, and finally popping the stackframe and
    returning the result. Compilers can optimize the code further and inline the
    function calls to eliminate the stackframe creation and the following
    procedures.
\end{sloppypar}
\begin{sloppypar}
    \citet{webdelays} studied the response times of web sites offered to the
    public. The websites response times where measured in seconds. These times
    can be viewed as being on the extreme. \citet{Johansson_2019} did an
    experiment on a monolithic and a microservices application. The recorded
    response times for the monolithic application where on average 64\% faster
    than a microservice application.
\end{sloppypar}
\begin{sloppypar}
    Requests sent to other microservices through a network are much slower than
    function calls within one computer. Therefore, the communication patterns
    should be changed to take into account the change in communication path. If
    the architecture is changed in such a way that the previous communication
    model among the components is preserved, there would be an excessive amount
    of communication and the resulting system would not be as performant
    \citep{fowlerlewisms}. 
\end{sloppypar}


% --------------------------------------------------------------------


\section{Security}
\begin{sloppypar}
    Implementing security is hard. A theoretical proof for this can be found in
    \citet{andersson2001information}. His main finding was that an attacker has
    an advantage over the developers trying to defend a system. For the system
    to be secure the defending developers have to find all of the bugs where as
    the attacker has to find only one. Therefore, to make systems more secure
    the use of tools and frameworks that have been tested and are already in use
    is preferable to developing one's own solution for a well known and already
    solved problem.
\end{sloppypar}
\subsection{Identifying Key Security Aspects}
\begin{sloppypar}
    As has been already established the communication within the web service
    differ greatly in MA and in MSA. Secure communication is one of the key
    security aspects to be discussed in this paper. In addition to the
    communication the Authentication and Authorization is the second critical
    security concern to be discussed. This will be addressed in the next
    chapter.
\end{sloppypar}

\section{Authentication and Authorization}
\begin{sloppypar}
    In the cases where the user has to be authenticated, the web service needs a
    way to do this securely. There are many authentication schemes available but
    users prefer the password \citep{passwordisdead}. Due to this the
    authentication is usually done using a tuple containing user credentials
    i.e. a username and a password. The user is authenticated and a key or token
    is transmitted to the user via the network. This communication in both MA
    and MSA, should be encrypted in a way that none of the actors in the
    transfer path can intercept the message and misuse the credentials. 
\end{sloppypar}
\begin{sloppypar}
    The credential counterparts i.e. the secret shared by the server and the
    user have to be available for the web service for verification. When using
    MSA, the service should own it's own data. When ever information is
    available it is a target for thieves and hackers. The services in MSA are to
    be individually deployable and the service scalable. Authentication service
    implementation has to take this into account. The service has to adhere to
    practices that minimize the risks of data breaches.
\end{sloppypar}
\begin{sloppypar}
    As has been mentioned previously implementing security is hard and resource
    intensive. Therefore, the authentication implementation should adhere to an
    already existing framework or an another entity providing the
    authentication. This is the case for both the MA and MSA. The available
    choices include Lightweight Directory Access Protocol (LDAP), OpenID Connect
    (OIDC), Security Assertion Markup Language (SAML), and Kerberos.
\end{sloppypar}

\subsection{Authentication and Authorization in MA}
\begin{sloppypar}
    In MA, it is possible to implement features in such ways that a process in
    which the application runs has access to a session or a user object or
    similar that carries user information. This information can consist of the
    granted roles and rights for the user. This information can be queried
    easily and securely when access control is needed.  to execute an action or
    operation or a specific authorization service or module can exist.
\end{sloppypar}
\begin{sloppypar}
    An example of an web service implemented in MA is presented in figure
    \ref{fig:MA_AUTH}. The user accessing the web service sends a request to
    load balancer which has all the information on the currently operational
    services. Each of the services run on a single process in which an
    authentication and authentication service or functionality are present. When
    a user is first accessing the service credentials are to be verified and in
    a successful case a session or a user object is created for the process.
\end{sloppypar}
\begin{figure}[h]
    \includegraphics[scale=0.70]{{MA_AUTH.png}}
    \centering
    \caption{Traditional load balanced MA web service \citep{authinmsa}}
    \label{fig:MA_AUTH}
\end{figure}

\subsection{Authentication and Authorization in MSA}
\begin{sloppypar}
    Authorization of the user rights can be implemented in various ways. One of
    which is an authorization service which can contain the access control
    matrix. Services being accessed verify from the authorization service that
    the client user or the role that the user has can access the requested
    service or functionality. In MA, the access rights to functionality can be
    implemented using annotations within the source code. The authorization is
    verified in memory and without any communication over the network.
\end{sloppypar}
\begin{figure}[h]
    \includegraphics[scale=0.75]{{MSA_AUTH_SERVICE.png}}
    \centering
    \caption{MSA authorization service \citep{authinmsa}}
    \label{fig:MSA_AUTH_SERVICE}
\end{figure}
\begin{sloppypar}
    In MSA accessing the access control matrix or matrices is not as easy as it
    is in MA. In order to verify that a specific right exists, the service would
    have communication with the authorization service. This communication would
    need to happen every time a user tries to access a functionality with access
    restrictions. This could potentially lead to extremely lively communication
    from all the services forming a bottleneck at to the authorization service.
    An example of such architecture is presented in figure
    \ref{fig:MSA_AUTH_SERVICE}.
\end{sloppypar}
\begin{figure}[h]
    \includegraphics[scale=0.75]{{TOKEN_AUTH.png}}
    \centering
    \caption{Bearer token based authentication \citep{authinmsa}}
    \label{fig:TOKEN_AUTH}
\end{figure}
\begin{sloppypar}
    To limit the excessive communication a token based authentication and
    authorization scheme can be used. A simplified process flow is presebted in
    \ref{fig:TOKEN_AUTH}. A user enters credentials to a login and with these
    credentials an authentication and authorization service grants the user a
    token. This token is used to access the services in the system. The services
    in question trust the issuer of the token and verify the token and the
    claims within. If the token is accepted and access can be granted the
    request is serviced without any communication between the services and the
    authorization service. A specific token based authentication protocol is
    discussed in the next chapter.
\end{sloppypar}

\subsection{OIDC}
\begin{sloppypar}
    OIDC is an identity layer to accompany OAuth 2.0 authorization framework
    based protocol. 
\end{sloppypar}
\begin{figure}[h]
    \includegraphics[scale=0.75]{{OIDC_FLOW.png}}
    \centering
    \caption{OIDC steps \citep{oidcflow}}
    \label{fig:oidcflow}
\end{figure}
\begin{sloppypar}
    An example of an OIDC flow is presented in figure \ref{fig:oidcflow}. The
    process flow steps are: the client application requests an authentication
    from the identity provider, the end user is authenticated and authorization
    is obtained, the identity provider responds to the initial authentication
    request by sending ID token and a possible access token, the client
    application can request the end user claims from the authorization server
    with the access token, and finally the authorization service returns the
    claims to the client application. The presented flow is similar to the OAuth
    2.0 flow and the token sent to the client application is a Java Script
    Object Notation Web Token (JWT).
\end{sloppypar}

\subsection{JWT}
\begin{sloppypar}
    JWT is a format to represent claims. It is base64 encoded, point separated
    strings, which can easily be carried in the HTTP request or response. The
    contents is key value pairs, and the token may or may not be signed and
    encrypted \citep{RFC7519}. The token may contain an expiration time. If the
    token is used to validate requests without a server side implementation that
    can revoke a token, it will be valid until this time. The JWT token is
    issued by an authority trusted by the service. In figure a basic bearer
    token based authentication and authorization is depicted. A client initially
    enters credentials to a login page and the authorization service issues a
    token for the user. This token is used in place of the credentials to access
    restricted resources.
\end{sloppypar}
\begin{sloppypar}
    The signing of JWT can be carried out in various ways. These are presented
    in the \citet{RFC7515}. The signature is computed using the algorithm and
    keys or certificates specified in the header values. When the token is
    signed using a private key it can be verified by all parties in possession
    of the public key.
\end{sloppypar}

\subsection{Opaque Token}
\begin{sloppypar}
    \citet{authinmsa} compared several authentication and authorization
    solutions. One of the solutions discussed in the paper was the usage of an
    opaque access token for the client and API Gateway communication and map an
    Opaque token to JWT in the API Gateway. This would be used in all other
    communication as the means to authenticate and authorize the user. The main
    problem the opaque token is said to solve is the logout problem. A bearer
    token is valid until expiration and the client can not invalidate the token.
    The opaque token flow is presented in figure
    \ref{fig:OPAQUE_TOKEN_API_GATEWAY}.
\end{sloppypar}
\begin{sloppypar}
    The use of two tokens has many advantages. The token granted to the client
    allows access only to the API Gateway. This token can not be used to access
    the services directly. Furthermore, the opaque token be can revoked by
    simply removing it from the storage for the token pairs. The usage of an
    internal JWT can alleviate some of the issues when MA is to be changed to
    MSA. Firstly, token theft though still possible, is limited to the time the
    client is logged on to the system. Secondly, though not advisable, the JWT
    can carry session information and the whole session. This allows for more of
    the previous MA implementation to be used without as many changes.
\end{sloppypar}
\begin{figure}[h]
    \includegraphics[scale=0.75]{{OPAQUE_TOKEN_API_GATEWAY.png}}
    \centering
    \caption{API Gateway and opaque token \citep{authinmsa}}
    \label{fig:OPAQUE_TOKEN_API_GATEWAY}
\end{figure}


% --------------------------------------------------------------------


\section{Communication}
\begin{sloppypar}
    As already explained, in MA, the service components can communicate using
    events, procedure calls or other methods available within a single server
    machine. Usually all this communication stays within a single computer and
    thus does not easily compromise confidentiality. This is not the case in
    MSA.
\end{sloppypar}
\begin{sloppypar}
    In MSA single services communicate via a network. There are multiple
    protocols or messaging system to choose from such as, REST API, Advanced
    Message Queuing Protocol (AMQP), Enterprise Service Bus (ESB), and Remote
    Procedure Calls (RPC). For a more complete list view \citet{secchalmsa}. In
    this paper REST API is to be examined further. In few instances some of the
    other systems are discussed briefly.
\end{sloppypar}
\begin{sloppypar}
    \citet{servicemesh} lists the typical fallacies that developers fall victim
    to when designing distributed systems: 
    \begin{itemize}
        \item The network is reliable
        \item Latency is zero
        \item Bandwidth is infinite
        \item The network is secure
        \item Topology doesn’t change
        \item There is one administrator
        \item Transport cost is zero
        \item The network is homogeneous
    \end{itemize}
\end{sloppypar}

\subsection{Representational State Transfer (REST)}
\begin{sloppypar}
    \citet{restroy} presented REST in 2000. REST has become a very successful
    architectural style. The style was derived using various constraints, one of
    which is stateless communication. This entails that a request must contain
    all the information needed to fulfill the request because the server does
    not keep track of the client. All session state is stored in the client, of
    which the server has no prior knowledge before a request. In her doctoral
    thesis, \citet{secchalmsa} critiques the REST paradigm from the security
    perspective. She states that the design of the architecture does not meet
    the security requirements for web applications. She also states that REST
    does not allow for any server side sessions and thus token revocation is
    impossible. Tokens can be validated only for the correct issuer by signature
    and for expiration. As such tokens are more compatible with REST, but there
    still has to be the public keys in the server for signature verification.
\end{sloppypar}

\subsection{Coping With Failure in Communication}
\begin{sloppypar}
    \citet{DBLP:journals/corr/MontesiW16} present widely used design pattern for
    MSA. The Circuit Breaker can be used to mitigate the very likely case that a
    microservice operates slower than the other services calling it and runs out
    of resources to fulfill the requests in time. The circuit breaker is either
    implemented in the microservice or as a proxy between the client and the
    microservice. When the microservice does not service requests as intended,
    the circuit breaker trips and sends a failure message to the clients
    immediately when requests are received, thus allowing the microservice time
    to service the prior requests.
\end{sloppypar}
\begin{sloppypar}
    The circuit breakers can prevent an application from becoming completely
    unresponsive and crashing when a denial of service attack is carried out on
    the service.
\end{sloppypar}


% --------------------------------------------------------------------


\section{Deployment Automation and Production}
\begin{sloppypar}
    Both MA and MSA web services can be installed on servers operated by the
    organization or individuals them selves. The software can be installed on
    the host operating system directly or a virtualization technology can be
    used.
\end{sloppypar}
\begin{sloppypar}
    \citet{closer18} implemented a test system mimicking the Deutsche Bahn seat
    reservation system using MSA. The purpose of the test system was to analyze
    security risks that were introduced by the implementation. In the study they
    categorized the solution to three layers: first of which was the compute
    provider, the second was the encapsulation technology, and the third one was
    the deployment. The technologies for these layers were: Amazon Web services,
    Docker for containers, and Kubernetes (k8s) nodes, respectively. They found
    out that the cloud-based infrastructure when used in MSA resulted in a more
    complex solution than in MA. The added layers such as the K8s, have to be
    configured correctly and an error in one could potentially compromise the
    whole system. In addition, implementation of security is very difficult and
    resource intensive. The rewards from a good security are invisible. When
    microservices are implemented or even planned the security should be taken
    into account as early as possible. Implementing security later in the
    project or as an after thought can be more expensive and very difficult.
\end{sloppypar}

\subsection{Virtualization}
\begin{sloppypar}
    Virtualization can be carried out in various ways but in this paper is to be
    on virtual machines (VM) and containerization. VM is a complete installation
    of all the software needed for a system to run. In its basic form a
    container uses the host operating system capabilities without the need to
    install operating system or non essential software a new. Only the
    application and its dependencies are needed. VMs are run on the host by a
    hypervisor and containers by an engine or the host operating system. Running
    an application on VM or on a container does not differ for the application
    or in this case, for the web service, the environment is similar.
\end{sloppypar}
\begin{sloppypar}
    Containers have considerably lower overhead when compared to VMs. A
    container can be created and started easily and automatically as needed by
    an orchestration solution such as, Docker Compose, Docker Swarm, and k8s.
    All of these tools have to be correctly configured and used according to
    their specification and best practice.
\end{sloppypar}

\subsection{Orchestration}
\begin{figure}[h]
    \includegraphics[scale=0.75]{{orchestration.png}}
    \centering
    \caption{Orchestration constituents\citep{containernetworking}}
    \label{fig:orchestration_constituents}
\end{figure}
\begin{sloppypar}
    Container orchestration can consist of the services and operations as is
    depicted in \ref{fig:orchestration_constituents}. The scaling refers to
    automatically creating or shutting down pods or containers to match
    utilization level. The containers and the microservices within can have a
    new version that needs to be rolled into the production. The upgrade service
    is responsible in doing this. Service discovery is a service which is used
    to locate running services and as a service to which self registration is
    carried out. If a service is to become non operational it might not be able
    to send the orchestrator any message on this erroneous behavior. Therefore,
    the orchestrator can have a Health check service. This service can
    periodically send a message to a service and if no response is not received
    with in reasonable time frame the service is deemed non operational and a
    new one is to be created by the scheduler. This service is to create the
    individual pods or containers according to the system settings. The last
    orchestration part is the organizational primitives. These are to used to
    e.g. label pods or containers with matching business domain names. This is
    to make the administration and setup work easier
    \citep{containernetworking}.
\end{sloppypar}

\subsection{Service Mesh}
\begin{figure}[h]
    \includegraphics[scale=0.75]{{service_mesh.png}}
    \centering
    \caption{Service mesh basic architecture \citep{servicemesh}}
    \label{fig:service_mesh_basic_architecture}
\end{figure}
\begin{sloppypar}
    A service mesh tries to solve the service-to-service communication
    challenges and to allow for monitoring of the entire system. In figure
    \ref{fig:service_mesh_basic_architecture} a very basic service mesh
    architecture is presented. The architecture consists of a Control plane and
    a Data plane. The Control plane offers: a user interface for system
    administrators, Policy Information Point (PIP) and Policy Decision Point
    (PDP), and collected metrics of the behavior and actions in and of the
    system. The data moves in the dataplane. Each microservice is accompanied by
    a proxy. All requests and responses flow through the proxies and are
    controlled by the Control plane \citep{servicemesh}. The microservices do
    not have to be aware of the proxy that is in the information path.
\end{sloppypar}
\begin{sloppypar}
    There are several products available for a service mesh, such as Istio,
    Linkerd, and Consul Connect to mention a few. As was the case with
    authentication and security in general it is not advisable to implement
    security critical features if there exists an off-the-shelf alternative.
\end{sloppypar}
\begin{sloppypar}
    The use of service mesh greatly simplifies the microservice implementation
    since the proxy can contain many of the features that would otherwise repeat
    in all of the services. The service mesh can provide a certificate authority
    for the communication between services. The proxies act on behalf of the
    service and all proxy-to-proxy communication can be encrypted on the
    transport layer using TLS \citep{pathtoservicemesh}. Furthermore, it
    facilitates the use of mTLS where both parties are verified. Service mesh
    also allows for multi cloud installations. 
\end{sloppypar}


% --------------------------------------------------------------------


\section{Other Security Concerns}
\subsection{Known Vulnerabilities}
\subsubsection{JWT}
\begin{sloppypar}
    The choices for the algorithm for signing the JWT algorithm contain "none"
    as one of the choices. This was found to be troublesome by . He found that
    many libraries did not operate in the desired way. The receiving party could
    be fooled to validate a mutated token without any signature with the "none"
    as it's algorithm. In addition to this vulnerability \citet{nonejwt} found
    that the verification suffered from another fatal flaw. When a token was
    created by using a symmetric algorithm, the servers could be fooled in to
    believing that a token signed by just the public key and not the secret HMAC
    key was a valid one.
\end{sloppypar}

\subsection{Transactions}
\begin{sloppypar}
    The data in the system is the resource to be protected. If this data becomes
    corrupted the system is not secure. In some cases the actions to be taken
    consists of multiple reads and writes of the data and the order of execution
    changes the end result.
\end{sloppypar}
\begin{sloppypar}
    Transactions can be used when updating database contents to make sure that
    atomicity, consistency, isolation, and durability (ACID) \citep{acid} is
    followed. When using MSA according to the definition each of the micro
    services should contain or have access to it's own data i.e. database. In MA
    Transactions are easier to implement especially when the execution is done
    sequentially. In MSA this is not necessarily the case. Performing an action
    might entail calling various microservices and if any of the individual
    actions fail permanentely the changes that have already been made need to be
    undone. In addition to this, a HTTP is asynchronious and the execution order
    can not be guaranteed. 
\end{sloppypar}
\begin{sloppypar}
    Transaction related issues can be solved by either, splitting the monolith
    in such a way that the actions that need to be carried out as transactions
    can be carried out in one microservice or by creating a service to
    coordinate the actions taken by the single microservices.
\end{sloppypar}

\subsection{Software Development}
\begin{sloppypar}
    When software is developed using MA, it is usually deployed as a whole and
    the program code can be compiled, tested and used as a single unit or
    multiple modules. In contrast, a service implemented in MSA can be deployed
    in single microservice units, and thus each component can be worked upon
    individually and deployed once ready. 
\end{sloppypar}
\begin{sloppypar}
    The immediacy in the deployment of the microservices entails a very specific
    security risk. \citet{integinside} present threats from malicious insiders
    working on the services as developers or other positions with access to
    sensitive information. In microservice development, the finished
    implementations are to be immediately released to production. There are
    steps in the CD pipeline prior to this but once tests pass in the test
    environments the pipeline is supposed to publish the changes to the actual
    production environment. The paper presents four specific threats. The first
    one is that the knowledge of sensitive information is spread among the
    developers more widely than in MA. This is due to access needs by
    developers. The second threat is that the insiders monitoring and operating
    the running system intentionally harm the system by making malicious
    changes. The third threat is the developers knowing the configurations and
    their ability to make almost instantaneous changes to them or the
    microservices themselves. The last presented threat in the paper is the
    non-repudiation. The system is not able to disallow malicious requests when
    the developers have had access to the keys and other configurations. They
    can effectively implement services or requests that emit malicious requests
    or responds. Malicious attempts in a MA are more easily screened by
    performing security audits and by peer reviewing the code. In a MSA the
    knowledge of a single service and it’s inner workings are shared by a more
    limited number of people. Finding the compromised actions from the
    interoperability of the distinct microservices is a daunting task.
\end{sloppypar}
\begin{sloppypar}
    Malicious attempts in a MA are more easily screened by performing security
    audits and by peer reviewing the code. In a MSA the knowledge of a single
    service and it's inner workings are shared by a more limited number of
    people. Finding the compromised actions from the interoperability of the
    distinct microservices is a daunting task.
\end{sloppypar}

\subsection{Externalized Configuration}
\begin{sloppypar}
    To allow for easy configuration change management there should exist a
    configuration orchestration service. This service should have an API from
    which services in their startup can load their appropriate configuration.
    The configuration of the whole system can be easily maintained through the
    API.
\end{sloppypar}
\begin{sloppypar}
    The contents of the configuration is highly sensitive information. It can
    consist of addresses, secrets, and other information that alter the behavior
    of the system. Secrets refer to credentials, connection strings, other keys
    and similar items that are to be kept confidential. Therefore, the content
    must be stored safely and not allowed to be read or altered by unauthorized
    users.
\end{sloppypar}

\subsection{Logging}
\begin{sloppypar}
    Logging in MA is relatively easy. The chosen logging solution is used and
    logs are created in easily configurable locations. In MSA this can be more
    difficult. Each of the microservices need to have their own logger and each
    of these need to be configured to log to a proper location. The logs that
    are created need to be persisted for the length of time allowed by
    legislation and according to the need of the system administration.
\end{sloppypar}
\begin{sloppypar}
    Without logs it is impossible to verify correct operation of the system nor
    is it possible to gain knowledge of a possible security breach. 
\end{sloppypar}


\subsection{Defense-in-Depth (DiD)}
\begin{sloppypar}
    It is not enough to secure the boundary between the perimeter and the
    internal system. In DiD the system should implement security measures at
    multiple layers within the system \citep{iec62443}. The system can be
    presented layered as in figure \ref{fig:Defense_in_depth}. Each of these
    layers should have security features and a breach in one should not
    compromise those features on an inner layer of the system.
\end{sloppypar}
\begin{sloppypar}
    \citet{moderndid} takes the idea of layered defense even further and
    proposes an integrated approach to DiD. He defines the defence layers as:
    edge routers, DDos defenses, Managed DNS, Reverse proxies, Bot Management,
    Web application firewalls, API defenses, and Caching. He further iterates
    that these layers or lines of defense should be aware of each other and be
    accessable from a single UI. Another approach he defines as human expertise.
    In this approach there exists an command center that is to be manned at all
    times and as such be able to respond to the encountered threats. In some
    instances this might be applicable, but for all web services this is not
    be feasible.
\end{sloppypar}
\begin{figure}[h]
    \includegraphics[scale=0.75]{{Defense_in_depth.png}}
    \centering
    \caption{Defense-in-Depth}
    \label{fig:Defense_in_depth}
\end{figure}


% --------------------------------------------------------------------


\section{Conclusion}
\begin{sloppypar}
    This paper discussed the security aspects of changing the architecture from
    MA to MSA.
\end{sloppypar}
\begin{sloppypar}
    In practice the previous MA that has been changed to MSA can have more
    aspects to go wrong than in previous MA implementation. The deployment can
    entail installing, virtualization, monitoring, and other tools. In some
    cases, these tools have to implemented by in house developers and thus more
    costs are incurred upfront and also in the upkeep of the system. In addition
    to being more costly own development has higher security risks involved.
\end{sloppypar}
\begin{sloppypar}
    MSA has higher complexity due to more tools needed and having more
    potentially exposed attack surface. Security can be though of as being as
    good as its weakest link. In general, a MSA deployment has multiple layers
    which all have to be consistent and correct. One example is the
    configuration of the operating system on the server running the
    virtualization environment. All of the layers from the server hardware to
    the handling of errors in the actual code have to be of ample quality to
    mitigate a failure in security.
\end{sloppypar}
\begin{sloppypar}
    The communication that was in monolith a simple in-process call might not be
    possible as such in a MSA web service. The individual services communicate
    via the network with high overhead in comparison to a simple function call.
    Furthermore, the identity and authorization of the entity requesting an
    action or data can usually be trusted in an in-process call. The mechanisms
    to allow for proper authentication and authorization amount to even higher
    overhead for the MSA. There exists a very real risk for the development team
    to implement an insufficient security scheme.
\end{sloppypar}
\begin{sloppypar}
    In MSA the security should be implemented in depth. There must be a healthy
    mistrust on all requests and security should be built in to the system.
\end{sloppypar}
\begin{sloppypar}
    Security has to be taken into account right from the beginning of the
    project in which the architecture is to be changed. The choices made in the
    development of the web service when following a MA do not carry to the MSA
    as such.
\end{sloppypar}
\begin{sloppypar}
    The use of pre-existing tools and frameworks is highly encouraged. The tools
    currently available such as Docker, k8s, and Istio solve many of the
    inherent security issues if applied correctly. The tools and frameworks that
    exist, can not be blindly trusted. As an example some of the implementations
    using JWT had and in some cases still have serious flaws and are not secure.
\end{sloppypar}
\begin{sloppypar}
    Authentication and authorization in MA and in MSA can differ greatly. In MA
    a session based authentication and authorization is applicable. This is not
    the case in MSA especially if a REST API is to be used. In REST API all the
    information needed to serve a request should be enclosed in the request.
    Tokens such as JWT cater this and can carry user information and other
    claims. The tokens are issued and signed by a trusted party. An API Gateway
    can act as an PEP and allow or disallow a request. Also, if a service mesh
    is used such as Istio this can be done in the proxy for a specific
    microservice.
\end{sloppypar}
\begin{sloppypar}
    In this paper many security aspects were not discussed. How do the many
    available design patterns for the splitting of the monolith fair when
    compared on security aspects. On the communication only REST API on HTTP was
    discussed in any real extent. 
\end{sloppypar}
\begin{sloppypar}
    The results found in this study can be used to determine if at all
    architecture change is feasible in a particular case. This can be the case
    after other architectural changes or code refactoring. In addition, the
    results can be used as a guideline for future research.
\end{sloppypar}
\begin{sloppypar}
    Further research is needed on the security aspects of the different design
    patterns that are available for the architectural change. Also, the
    implementation of a field level access control should be studied further.
\end{sloppypar}

% Loppuluku päättää työn. Luvun nimi on tyypillisesti ``yhteenveto'' tai
% ``johtopäätöksiä''. Valitse se otsikko, joka tuntuu sopivammalta työsi
% luonteeseen. Joka tapauksessa loppuluku sisältää niin työn yhteenvedon kuin
% johtopäätöksiä työn tulosten perusteella. Pääajatus on antaa lukijalle selvä
% kuva siitä, miten johdannossa asetettuihin tavoitteisiin työssä vastattiin.

% Käsittele loppupuvussa seuraavia asioita (jotakuinkin tässä järjestyksessä):
% %
% \begin{itemize} \item Muistutus työn tavoitteista (sidoksisuus johdantoon)
%   \item Päätulokset kootaan yhteen, pohditaan niiden merkitystä \item
%   Suositukset konkreettisiksi toimenpiteiksi (``Mitä sitten?'' Nyt kun
%   käytössä on tämän työn myötä tullut tieto, mitä se nyt tarkoittaa tälle
%   asialle/alalle.) \item Tulosten soveltuvuus, käyttöön liittyvät rajoitukset
%   \item Jatkotutkimustarve (``Tulevaisuudessa olisi mielenkiintoista
%   selvittää...'' tms.) \item Työn onnistumisen arviointi (Huom! Älä arvioi
%   omaa kirjoitusprosessiasi vaan tekemääsi tutkimusta) \end{itemize}


% --Junk starts here--------------------------------------------------
% --------------------------------------------------------------------
%     \begin{sloppypar} authorization, access control, policy decision point (PDP),
%         policy enforcement point (PEP), A Framework for Policy-based Admission
%         Control https://tools.ietf.org/html/rfc2753 \end{sloppypar}
% --------------------------------------------------------------------
% --------------------------------------------------------------------
%     \begin{sloppypar} \end{sloppypar} \begin{sloppypar}
%     https://www.nginx.com/blog/service-discovery-in-a-microservices-architecture/
%     https://www.consul.io \end{sloppypar}
% --------------------------------------------------------------------
% --------------------------------------------------------------------
%     \section{Orchestration solutions} \begin{sloppypar} Docker Swarm, Kubernetes
%     (K8s), Azure, sandbox, virtualization \end{sloppypar}
% --------------------------------------------------------------------
% --------------------------------------------------------------------
%     \section{Authorization} \begin{sloppypar}
%     https://techbeacon.com/security/microservices-apps-do-identity-access-management-without-overhead
%     \end{sloppypar} \begin{sloppypar} token (JWT, OAuth 2.0), (API gateway, IdP)
%     vs distribution \end{sloppypar} \begin{sloppypar} Identity and Access
%     Management (IAM), Identity federation, Azure Active Directory, Active
%     Directory Federation Services (ADFS), Google, Facebook, Twitter, LinkedIn
%     etc., Single-Sign-On (SSO), Security Assertion Markup Language (SAML), SAML
%     identity provider, OpenId provider, OAuth 2.0, OpenId Connect (OIDC), Identity
%     Management: Identification, Authentication, Authorization. \dots
%     \end{sloppypar}
% --------------------------------------------------------------------
% --------------------------------------------------------------------
%         TODO
%         Next messaging systems list is from \citet{secchalmsa}:
%     lightweight REST API, Sync RPC, GraphQl Async REST, gRPC Apache Kafka, ZeroMQ
%         Java Message Service: ActiveMQ, JBOSS messaging, Glassfish AMQP: RabbitMQ,
%         Qpid, HornetQ MuleESB, Apache ServiceMix, JBossESB heavyweight WebSocket
% --------------------------------------------------------------------
% --------------------------------------------------------------------
%     \subsection{Event-Driven Communication} \begin{sloppypar} \end{sloppypar}
% --------------------------------------------------------------------
% --------------------------------------------------------------------
%     \section{Decision Flowcharts} \begin{sloppypar} In this chapter flowcharts are
%     presented that capture the main findings and the factors affecting decision
%     making in and architectural switch. \end{sloppypar} \begin{sloppypar} In
%     figure \ref{fig:auth_flowchart} a decision flowchart for authentication and
%     authorization is presented. \end{sloppypar} \begin{figure}[h]
%     \includegraphics[scale=0.75]{{AUTH_FLOWCHART.png}} \centering
%     \caption{Authorization flowchart.} \label{fig:auth_flowchart} \end{figure}
% --------------------------------------------------------------------
% --------------------------------------------------------------------
%     The amount
%     of code to refactor is much larger than in a small microservice.
%     Microservice should do one thing and, as such, it should be more
%     understandable.
% --------------------------------------------------------------------
% --------------------------------------------------------------------
% \section{Session} \begin{sloppypar} Client handled session is easy to tamper
% with and has risks involved. The service would either have to trust the client
% offered session or verify a signature. In both of these cases the client
% application would have to be aware of the service and would be able to
% communicate with it directly. This would bring considerable overhead on all
% the services and a security risk. \end{sloppypar} \begin{sloppypar} Sessions
% can be used in MSA when the architecture has an API Gateway. The session is
% stored in the Gateway and a session key is carried in the requests made by the
% client application. The API Gateway then verifies the request and can grant
% access to specific service. Without the Gateway the session would have to be
% either centrally maintained at a session service or be handled by the client
% and passed along in the requests. \end{sloppypar}
% --------------------------------------------------------------------
% --------------------------------------------------------------------
% \subsubsection{Scaling} \begin{sloppypar} \end{sloppypar}
% \subsubsection{Upgrading} \begin{sloppypar} \end{sloppypar}
% \subsubsection{Service Discovery} \begin{sloppypar} Service discovery as
% presented in \citet{DBLP:journals/corr/MontesiW16} is a design pattern in
% which a registry is kept on currently running microservices. The microservices
% register them selves to the service discovery registry. This registry is used
% by either a router to route client service calls to running microservices or
% by the client directly. \end{sloppypar}
% --------------------------------------------------------------------
% --------------------------------------------------------------------
% \subsection{Testing} \begin{sloppypar} The possible ways to fail for a web
% service consisting of a large number of microservices are numerous. One
% solution for testing for outages is the Netflix created Chaos Monkey
% \citep{chaosmonkey}. It is used for resilience testing and can terminate
% containers or VMs at random. \end{sloppypar}
